\documentclass[paper.tex]{subfiles}

\usepackage{tikz}
\usepackage{amsmath}
\usepackage{amssymb}
\usepackage{graphicx}
\usepackage{tabularx}
\usepackage{multicol}
\usepackage{algpseudocode}
\usepackage{algorithm}

% Add vertical spacing to tables
\renewcommand{\arraystretch}{1.4}

% Begin Document
\begin{document}

\section{$LU$ Factorization}

The first method to solve this system of equations will be $LU$ Factorization.
This involves decomposing the coefficient matrix $A$ into two factors, $L$ and $U$, resulting in an equation of the form $A = LU$.
Doing so gives:
\begin{center}
\begin{equation*}
    A = \begin{bmatrix}
        1 & 0 & 0 & 0 & 0 \\
        \frac{1}{2} & 1 & 0 & 0 & 0 \\
        \frac{1}{2} & -1 & 1 & 0 & 0 \\
        0 & -2 & 1 & 1 & 0 \\
        0 & 2 & -\frac{3}{2} & -1 & 1
    \end{bmatrix}
    \begin{bmatrix}
        2 & 1 & 0 & 0 & 0 \\
        0 & \frac{1}{2} & -1 & 0 & -1 \\
        0 & 0 & -2 & 0 & -2 \\
        0 & 0 & 0 & 1 & 1 \\
        0 & 0 & 0 & 0 & 1 
    \end{bmatrix}
\end{equation*}
\end{center}

Using the $LU$ decomposition it's possible to decompose the original matrix equation into two separate equations:
\begin{equation}
    Ux = y
\end{equation}
\begin{equation}
    Ly = b
\end{equation}

We must first solve for $y$ in equation (3). 
\begin{equation}
    y = L^{-1}b
\end{equation}

This requires the inverse of $L$, which is:
\begin{equation*}
    \begin{bmatrix}
        1 & 0 & 0 & 0 & 0 \\
        -\frac{1}}{2} & 1 & 0 & 0 & 0 \\
        -1 & 1 & 1 & 0 & 0 \\
        0 & 1 & -1 & 1 & 0 \\
        -\frac{1}}{2} & \frac{1}}{2} & -\frac{1}}{2} & 1 & 1 
    \end{bmatrix}
\end{equation*}

Plugging in values for equation (4) gives:
\begin{equation*}
    y = \begin{bmatrix}
        1 & 0 & 0 & 0 & 0 \\
        -\frac{1}}{2} & 1 & 0 & 0 & 0 \\
        -1 & 1 & 1 & 0 & 0 \\
        0 & 1 & -1 & 1 & 0 \\
        -\frac{1}}{2} & \frac{1}}{2} & -\frac{1}}{2} & 1 & 1
    \end{bmatrix}
    \begin{bmatrix}
        100 \\
        0 \\
        -50 \\
        120 \\
        0
    \end{bmatrix}
    =\begin{bmatrix}
        100 \\
        -50 \\
        -150 \\
        50 \\
        95 \\
    \end{bmatrix}
\end{equation*}

In order to compute for $x$, we rearrange equation (2) into:
\begin{equation}
    x = U^{-1}y
\end{equation}

\end{document}

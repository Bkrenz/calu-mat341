\documentclass[paper.tex]{subfiles}

\usepackage{tikz}
\usepackage{amsmath}
\usepackage{amssymb}
\usepackage{graphicx}
\usepackage{tabularx}
\usepackage{multicol}
\usepackage{algpseudocode}
\usepackage{algorithm}

% Add vertical spacing to tables
\renewcommand{\arraystretch}{1.4}

% Begin Document
\begin{document}

\section{A System of Linear Equations}

This problem can be solved by utilizing a system of linear equations.
Each equation represents the input and output of a specific router node.
First we must construct a table of the input and output flow of each node.
The input and output must be equivalent for each node.
Solving for variables for each node gives us a system of linear equations.

\begin{center}
\begin{tabular}{c | l | l}

    \textbf{Node} & \textbf{Input} & \textbf{Output} \\ \hline

    A & 100                         & $2x_1 + x_2$ \\
    B & $x_1 + x_2$                 & $x_3 + x_5$  \\
    C & $50 + x_1$                  & $x_3 + x_5$  \\
    D & $x_4 + x_5$                 & $x_2 + 120$  \\
    E & $x_2 + x_3 + x_5$ \hspace{8mm} & $x_4$                         
    
\end{tabular}


\begin{tabular}{c | l}
    \textbf{Node} & \textbf{Equations} \\ \hline
    A & $2x_1 + x_2 = 100$ \\
    B & $x_1 + x_2 - x_3 - x_5 = 0$    \\
    C & $x_1 - x_3 - x_5 = -50$        \\
    D & $-x_2 + x_4 + x_5 = 120$       \\
    E & $x_2 + x_3 - x_4 + x_5 = 0$
\end{tabular}
\end{center}

Utilizing this system of equations, we can construct a matrix equation of the form $Ax = b$ representing it.
This becomes equation (1) below:

\begin{center}
\begin{equation}
    \begin{bmatrix}
        2 & 1 & 0 & 0 & 0   \\
        1 & 1 & -1 & 0 & -1 \\
        1 & 0 & -1 & 0 & -1 \\
        0 & -1 & 0 & 1 & 1  \\
        0 & 1 & 1 & -1 & 1 
    \end{bmatrix}
    \begin{bmatrix}
        x_1 \\
        x_2 \\
        x_3 \\
        x_4 \\
        x_5
    \end{bmatrix}
    =
    \begin{bmatrix}
        100 \\
        0 \\
        -50 \\
        120 \\
        0
    \end{bmatrix}
\end{equation}
\end{center}

\end{document}

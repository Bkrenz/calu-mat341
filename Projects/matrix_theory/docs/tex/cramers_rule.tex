\documentclass[paper.tex]{subfiles}

\usepackage{tikz}
\usepackage{amsmath}
\usepackage{amssymb}
\usepackage{graphicx}
\usepackage{tabularx}
\usepackage{multicol}
\usepackage{algpseudocode}
\usepackage{algorithm}

% Add vertical spacing to tables
\renewcommand{\arraystretch}{1.4}

% Begin Document
\begin{document}

\section{Cramer's Rule}

The second method used to solve this is the use of Cramer's Rule.
Let us first recap the original matrix equation.
\begin{equation*}
    \begin{bmatrix}
        2 & 1 & 0 & 0 & 0   \\
        1 & 1 & -1 & 0 & -1 \\
        1 & 0 & -1 & 0 & -1 \\
        0 & -1 & 0 & 1 & 1  \\
        0 & 1 & 1 & -1 & 1 
    \end{bmatrix}
    \begin{bmatrix}
        x_1 \\
        x_2 \\
        x_3 \\
        x_4 \\
        x_5
    \end{bmatrix}
    =
    \begin{bmatrix}
        100 \\
        0 \\
        -50 \\
        120 \\
        0
    \end{bmatrix}
\end{equation*}

We must first find the determinant of $A$, as noted below:
\begin{equation*}
    \begin{vmatrix}
        A
    \end{vmatrix}
    = \begin{vmatrix}
        2 & 1 & 0 & 0 & 0   \\
        1 & 1 & -1 & 0 & -1 \\
        1 & 0 & -1 & 0 & -1 \\
        0 & -1 & 0 & 1 & 1  \\
        0 & 1 & 1 & -1 & 1 
    \end{vmatrix}
    = -2
\end{equation*}

Following Cramer's rule then leads to finding the determinant of each $A_i(b)$, which indicates the matrix formed when replacing the $i^{th}$ column of $A$ with $b$.
\begin{align*}
    &\begin{vmatrix}A_1\end{vmatrix} = \begin{vmatrix}
        100 & 1 & 0 & 0 & 0   \\
        0 & 1 & -1 & 0 & -1 \\
        -50 & 0 & -1 & 0 & -1 \\
        120 & -1 & 0 & 1 & 1  \\
        0 & 1 & 1 & -1 & 1
    &\end{vmatrix} = -50\\
    &\begin{vmatrix}A_2\end{vmatrix} = -100 \\ 
    &\begin{vmatrix}A_3\end{vmatrix} = -60 \\ 
    &\begin{vmatrix}A_4\end{vmatrix} = -250 \\
    &\begin{vmatrix}A_5\end{vmatrix} = -90 \\
\end{align*}

Then for each determinant, division by the original determinant provides the solution to that particular $x$ value.
\begin{equation*}
    \frac{1}{-2} \begin{bmatrix}
        -50 \\
        -100 \\
        -60 \\
        -250 \\
        -90
    \end{bmatrix}
\end{equation*}

The result is then the answer:
\begin{equation*}
    x = \begin{bmatrix}
        25 \\
        50 \\ 
        30 \\ 
        125 \\
        45
    \end{bmatrix}
\end{equation*}

This answer is identical to the answer obtained through $LU$ Factorization.

\end{document}

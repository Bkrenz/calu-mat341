\documentclass[paper.tex]{subfiles}

\usepackage{tikz}
\usetikzlibrary{arrows.meta}

\usepackage{amsmath}
\usepackage{graphicx}
\usepackage{tabularx}
\usepackage{multicol}
\usepackage{algpseudocode}
\usepackage{algorithm}

% Add vertical spacing to tables
\renewcommand{\arraystretch}{1.4}

% Begin Document
\begin{document}

\section{Introduction}

This project demonstrates Matrix Theory when applied to a network throughput graph.
The scenario to be solved is as follows.

A sender is transmitting data with a total rate of 150 megabits per second along the network graph in Figure 1.
The data is transmitted from the sender to the receiver over a network of five different routers.
These routers are labeled A, B, C, D, and E. The connections and data rates between the routers are labeled as $x_1, x_2, x_3, x_4, x_5$.
This project will find a solution to the data transfer through the specific routers in this network.
Two methods will be used to find the solution: $LU$ Factorization and Cramer's Rule.
At the end will be a recommendation on network changes.

\end{document}
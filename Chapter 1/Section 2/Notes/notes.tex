\documentclass{article}

\usepackage{amsmath}
\usepackage{graphicx}
\usepackage{tabularx}
\usepackage{multicol}

\usepackage[english]{babel}
\newtheorem{theorem}{Theorem}

% Geometry 
\usepackage{geometry}
\geometry{letterpaper, left=15mm, top=20mm, right=15mm, bottom=20mm}

% Fancy Header
\usepackage{fancyhdr}
\renewcommand{\footrulewidth}{0.4pt}
\pagestyle{fancy}
\fancyhf{}
\chead{MAT 341 - Linear Algebra}
\lfoot{CALU Fall 2021}
\rfoot{RDK}

% Add vertical spacing to tables
\renewcommand{\arraystretch}{1.4}

% Macros
\newcommand{\definition}[1]{\underline{\textbf{#1}}}

\newenvironment{rcases}
  {\left.\begin{aligned}}
  {\end{aligned}\right\rbrace}

% Begin Document
\begin{document}

\section*{Section 1.2: Row Reduction and Echelon Forms}

\begin{itemize}

\item A rectangular matrix is in \definition{row echelon form} if it has the following three properties:
\begin{enumerate}
  \item All nonzero rows are above any rows of all zeros.
  \item Each leading entry of a row is in a column to the right of the leading entry of the row above it.
  \item All entries in a column below a leading entry are zeros.
\end{enumerate}

\item If a matrix in echelon form satisfies the following additional conditions, then it is \definition{reduced row echelon form}:
\begin{enumerate}
  \item The leading entry in each nonzero row is 1.
  \item Each leading 1 is the only nonzero entry in its column.
\end{enumerate}

\item An \definition{echelon matrix} (respectively, \definition{reduced echelon matrix}) is one that is in echelon form (respectively, reduced echelon form).

\item Any nonzero matrix may be \definition{row reduced} (ie, transformed by elementary row operations) into more than one matrix in echelon form, using different sequences of row operations.
However, the reduced echelon form one obtains from a matrix is unique.

\end{itemize}

\noindent\fbox{
  \parbox{\textwidth}{
    \begin{theorem}[Uniqueness of the Reduced Echelon Form]
      Each matrix is row equivalent to one and only one reduced echelon matrix.
    \end{theorem}
  }
}

\begin{itemize}

  \item If a matrix $A$ is row equivalent to an echelon matrix $U$, we can $U$ an \textbf{echelon form of $A$}; if $U$ is in reduced echelon form, we call $U$ the \textbf{reduced echelon form of $A$}.

  \item A \definition{pivot position} in a matrix $A$ is a location in $A$ that corresponds to a leading $1$ in the reduced echelon form of $A$. A \definition{pivot column} is a column of $A$ that contains a pivot position.

  \item \textbf{Using Row Reduction to Solve a Linear System}
  \begin{enumerate}
    \item Write the augmented matrix of the system.
    \item Use the row reduction algorithm to obtain an equivalent augmented matrix in echelon form. Decide whether the system is consistent. If there is no solution, stop; otherwise, go to the next step.
    \item Continue row reduction to obtain the reduced echelon form.
    \item Write the system of equations corresponding to the matrix obtained in step 3.
    \item Rewrite each nonzero equation from step 4 so that its one basic variable is expressed in terms of any free variables appearing in the equation.
  \end{enumerate}

\end{itemize}

\end{document}
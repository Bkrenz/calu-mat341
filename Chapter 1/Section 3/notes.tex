\documentclass{article}

\usepackage{amsmath}
\usepackage{amssymb}

\usepackage{graphicx}
\usepackage{tabularx}
\usepackage{multicol}

\usepackage[english]{babel}
\newtheorem{theorem}{Theorem}

% Geometry 
\usepackage{geometry}
\geometry{letterpaper, left=15mm, top=20mm, right=15mm, bottom=20mm}

% Fancy Header
\usepackage{fancyhdr}
\renewcommand{\footrulewidth}{0.4pt}
\pagestyle{fancy}
\fancyhf{}
\chead{MAT 341 - Linear Algebra}
\lfoot{CALU Fall 2021}
\rfoot{RDK}

% Add vertical spacing to tables
\renewcommand{\arraystretch}{1.4}

% Macros
\newcommand{\definition}[1]{\underline{\textbf{#1}}}

\newenvironment{rcases}
  {\left.\begin{aligned}}
  {\end{aligned}\right\rbrace}

% Begin Document
\begin{document}

\section*{Section 1.3: Vector Equations}

\begin{itemize}

  \item A matrix with only one column is called a \definition{column vector}, or simply a \definition{vector}.
  
  \item An example of a vector with two entries, where $w_1$ and $w_2$ are any real numbers, is:
  \begin{equation*}
      w = \begin{bmatrix}
        w_1 \\ w_2
      \end{bmatrix}
  \end{equation*}

  \item The set of all vectors with 2 entries is denoted by $\mathbb{R}^2$.
  
  \item The $\mathbb{R}$ stands for the real numbers that appear as entries in the vector, and the exponent 2 indicates that each vector contains 2 entries.
  
  \item Two vectors in $\mathbb{R}^2$ are \textbf{equal} if and only if their corresponding entries are equal.
  
  \item Given two vectors \textbf{u} and \textbf{v} in $\mathbb{R}^2$, their \textbf{sum} is the vector $\textbf{u} + \textbf{v}$ obtained by adding the corresponding entries of \textbf{u} and \textbf{v}.
  
  \item Given a vector \textbf{u} and a real number \textbf{c}, the \definition{scalar multiplication} of \textbf{u} by \textbf{c} is the vector \textbf{cu} obtained by multiplying each entry in \textbf{u} by \textbf{c}.
  
  \item Consider a rectangular coordinate system in the plane. Because each point in the plane is determined by an ordered pair of numbers, we can identify a geometric point $(a,b)$ with the column vector 
  $\begin{bmatrix}
    a \\ b
  \end{bmatrix}$.

\end{itemize}

\noindent\fbox{
  \parbox{\textwidth}{
    \begin{theorem}[Existence and Uniqueness Theorem]
      A linear system is consistent if and only if the rightmost column of the augmented matrix is \textit{not} a pivot column; ie, if and only if an echelon form of the augmented matrix has no row of the form $[0 \cdots 0 b]$ with $b$ nonzero.
    \end{theorem}
  }
}

\end{document}
\documentclass{article}

\usepackage{amsmath}
\usepackage{amssymb}

\usepackage{graphicx}
\usepackage{tabularx}
\usepackage{multicol}
\usepackage{enumitem}

\usepackage[english]{babel}
\newtheorem{theorem}{Theorem}

% Geometry 
\usepackage{geometry}
\geometry{letterpaper, left=15mm, top=20mm, right=15mm, bottom=20mm}

% Fancy Header
\usepackage{fancyhdr}
\renewcommand{\footrulewidth}{0.4pt}
\pagestyle{fancy}
\fancyhf{}
\chead{MAT 341 - Linear Algebra}
\lfoot{CALU Fall 2021}
\rfoot{RDK}

% Add vertical spacing to tables
\renewcommand{\arraystretch}{1.4}

% Macros
\newcommand{\definition}[1]{\underline{\textbf{#1}}}

\newenvironment{rcases}
  {\left.\begin{aligned}}
  {\end{aligned}\right\rbrace}

% Begin Document
\begin{document}

\section*{Section 1.4: The Matrix Equation}

\begin{itemize}

  \item If $A$ is an $m \times n$, with columns $a_1, \ldots, a_n$, and if $x$ is in $\mathbb{R}^n$, then the product of $A$ and $x$, denoted by $Ax$, is the linear combination
  of the columns of $A$ using the corresponding entries in $x$ as weights; that is:
  \begin{equation*}
    Ax = 
    \begin{bmatrix}
      a_1 & a_2 & \cdots & a_n
    \end{bmatrix}
    \begin{bmatrix}
      x_1 \\ x_2 \\ \cdots \\ x_n
    \end{bmatrix}
    = x_1a_1 + x_2a_2 + \cdots + x_na_n
  \end{equation*}

  \item $Ax$ is defined only if the number of columns of $A$ equals the number of entries in $x$.

\end{itemize}

\noindent\fbox{
  \parbox{\textwidth}{
    \begin{theorem}
      If $A$ is an $m \times n$ matrix, with columns $a_1, \ldots, a_n$, and if $b$ is in $\mathbb{R}^m$, then the matrix equation $Ax = b$ has the same solution set as the vector equation
      \begin{equation*}
        x_1a_1 + x_2a_2 + \cdots + x_na_n
      \end{equation*}
      which, in turn, has the same solution set as the system of linear equations whose augmented matrix is 
      \begin{equation*}
        Ax = \begin{bmatrix}
          a_1 & a_2 & \ldots & a_n & b
        \end{bmatrix}
      \end{equation*}
    \end{theorem}
  }
}

\begin{itemize}

  \item The equation $Ax = b$ has a solution if and only if $b$ is a linear combination of the columns of $A$.

\end{itemize}

\noindent\fbox{
  \parbox{\textwidth}{
    \begin{theorem}
      Let $A$ be an $m \times n$ matrix. Then the following statements are logically equivalent. That is, for a particular $A$, either they are all true statements or they are all false.
      \begin{enumerate}[label=\alph*.)]
        \item For each $b$ in $\mathbb{R}^m$, the equation $Ax = b$ has a solution.
        \item Each $b$ in $\mathbb{R}^m$ is a linear combination of the columns of $A$.
        \item The columns of $A$ span $\mathbb{R}^m$.
        \item $A$ has a pivot position in every row.
      \end{enumerate}
    \end{theorem}
  }
}

\begin{itemize}

  \item The matrix with $1$s on the diagonal and $0$s elsewhere is called the \definition{identity matrix} and is denoted by $I$:
  \begin{equation*}
    \begin{bmatrix}
      1 & 0 & 0 \\
      0 & 1 & 0 \\
      0 & 0 & 1
    \end{bmatrix}
  \end{equation*}

\end{itemize}

\noindent\fbox{
  \parbox{\textwidth}{
    \begin{theorem}
      If $A$ is an $m \times n$ matrix, $u$ and $v$ are vectors in $\mathbb{R}^n$, and $c$ is a scalar, then 
      \begin{enumerate}[label=\alph*.)]
        \item $A(u + v) = Au + Av$
        \item $A(cu) = c(Au)$
      \end{enumerate}
    \end{theorem}
  }
}



\end{document}